To quantify the effectiveness of this method we compare it to the existing OpenDataCam tool.
We will compare the counts, produced by OpenDataCam and manual annotation, of cyclists at strategic parts of the intersection with the counts produced by our system. The manually annotated counts being the ground truth.
This is followed by an analysis of the desire paths and an analysis of body language when cyclist trigger a defined
alert zone.
\ \\ 

\subsubsection{Comparison between three different methods}
Table x shows a comparison between OpenDataCam, manual annotation and our method.
The comparison is based on how well each method captured cyclist driving on along three different 
desire lines towards Fisketorvet.
Our method captured x ‰ of what manual annotation has, manual annotation being the base line.
\ \\

\subsubsection{Interpretation of desire paths desire paths}
The trajectory analysis in Figure~\ref{Rainbow} we can determine eight commonly taken desire paths. 
Where as with manual annotation we can determine x amount.
\ \\
\raggedbottom
\ \\ 
\noindent
\begin{tabular}{@{}cc}
\includegraphics[width=1.0\columnwidth]{desire_paths_overhead} 
\end{tabular}
\captionof{figure}{Common trajectories of cyclists comming from NW}
\label{traject}

\ \\
\subsubsection{behavioral analysis}
As mentinoed in the web application, we couple the the top-level view with references to time stamps in the recorded footage, 
such that the behavior of cyclists can be observed at street level.
Whenever a unique trajectory is selected or a part of the intersection is marked off, one can inspect each individual cyclist 
and the context leading up to a detection.
\\

In the below image we see a typical detection of a rider passing straight over Dybbølsbro instead of turning left and 
continuing over the bi-directional cycle path. Uknowingly thinking he is in the right, he closely passes, at full speed, another rider crossing
the street. Shortly after he notices the traffic markings and turns around. 

\ \\ 
\raggedbottom
\begin{tabular}{@{}cc}
\includegraphics[width=1.0\columnwidth]{behaviour_fast} 
\end{tabular}
\captionof{figure}{Detection zone triggered}
\label{Alert}
\ \\

Another commonly observed behaviour was cyclists trying to shorten the trajectories of their turns 
at the "waiting corner" at Ingerslevgade.

\raggedbottom
\ \\ 
\noindent
\begin{tabular}{@{}cc}
\includegraphics[width=1.0\columnwidth]{shorten_traj} 
\end{tabular}
\captionof{figure}{Shortend turn}
\label{Alert}
\ \\

When volumes of other traffic was low or over the short window of all traffc having red lights, cyclists would use the opertunity 
to switly cross the intersection while still "following the correct path" (that is, not crossing in a fully diagnol path). 

% Which are all compared to the "ground truth" dataset that we annotate manually. We know from previous works with OpenDataCam
% that vehicle detection has achieved 95\% accuracy while performing worse for pedestrians and motorcycles (\cite{BROEKMAN2021100068}).

