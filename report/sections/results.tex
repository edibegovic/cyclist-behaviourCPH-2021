Here we expect to cover a mix of some technical metrics of the system as well as the \textit{case study} on Dybbølsbro. 
For the former, we expect cover the precision of OpenDataCam (including YOLO4) with stats such as

\begin{itemize}
	\item Share of bicycles detected (\%)
	\item Mean and median trajectory length
	\item For all frames a bike is visible (and detected at least once), what share of them was the bike detected.
\end{itemize}

which are all compared to "ground truth" dataset that we annotate manually. We know from previoues works with OpenDataCam
that vehecle detection has achieved 95\% accuracy, while performing worse for pedestrians and motorcycles (\cite{BROEKMAN2021100068}).

\ \\
On the Dybbølsbro study we will focus on how well it reproduces the real-world trajectories (desire lines) that
we observe. We expect to 


\raggedbottom
\ \\ 
\noindent
\begin{tabular}{@{}cc}
\includegraphics[width=1.0\columnwidth]{paths} 
\end{tabular}
\captionof{figure}{Common trajectories of cyclists comming from NW}