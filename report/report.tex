%%%%%%%%%%%%%%%%%%%%%%%%%%%%%%%%%%%%%%%%%
% Wenneker Article
% LaTeX Template
% Version 2.0 (28/2/17)
%
% This template was downloaded from:
% http://www.LaTeXTemplates.com
%
% License:
% CC BY-NC-SA 3.0 (http://creativecommons.org/licenses/by-nc-sa/3.0/)
%
%%%%%%%%%%%%%%%%%%%%%%%%%%%%%%%%%%%%%%%%%

%----------------------------------------------------------------------------------------
%	PACKAGES AND OTHER DOCUMENT CONFIGURATIONS
%----------------------------------------------------------------------------------------

\documentclass[10pt, a4paper, twocolumn]{article} % 10pt font size (11 and 12 also possible), A4 paper (letterpaper for US letter) and two column layout (remove for one column)

%%%%%%%%%%%%%%%%%%%%%%%%%%%%%%%%%%%%%%%%%
% Arsclassica Article
% Structure Specification File
%
% This file has been downloaded from:
% http://www.LaTeXTemplates.com
%
% Original author:
% Lorenzo Pantieri (http://www.lorenzopantieri.net) with extensive modifications by:
% Vel (vel@latextemplates.com)
%
% License:
% CC BY-NC-SA 3.0 (http://creativecommons.org/licenses/by-nc-sa/3.0/)
%
%%%%%%%%%%%%%%%%%%%%%%%%%%%%%%%%%%%%%%%%%

%----------------------------------------------------------------------------------------
%	REQUIRED PACKAGES
%----------------------------------------------------------------------------------------

\usepackage[
%nochapters, % Turn off chapters since this is an article        
beramono, % Use the Bera Mono font for monospaced text (\texttt)
eulermath,% Use the Euler font for mathematics
pdfspacing, % Makes use of pdftex letter spacing capabilities via the microtype package
dottedtoc % Dotted lines leading to the page numbers in the table of contents
]{classicthesis} % The layout is based on the Classic Thesis style


\usepackage{arsclassica} % Modifies the Classic Thesis package

\usepackage[T1]{fontenc} % Use 8-bit encoding that has 256 glyphs

\usepackage[utf8]{inputenc} % Required for including letters with accents

\usepackage{graphicx} % Required for including images

\usepackage{enumitem} % Required for manipulating the whitespace between and within lists

\usepackage{lipsum} % Used for inserting dummy 'Lorem ipsum' text into the template

\usepackage{subfig} % Required for creating figures with multiple parts (subfigures)

\usepackage{amsmath,amssymb,amsthm} % For including math equations, theorems, symbols, etc

\usepackage{varioref} % More descriptive referencing
\usepackage{color}
\usepackage{listings}
\usepackage{setspace}
\usepackage{lscape}
\usepackage[authoryear]{natbib}

%----------------------------------------------------------------------------------------
%	DUTCH SUPPORT
%----------------------------------------------------------------------------------------

%\usepackage[dutch]{babel}    % comment out if you write your thesis in Dutch


%----------------------------------------------------------------------------------------
%	THEOREM STYLES
%---------------------------------------------------------------------------------------

\theoremstyle{definition} % Define theorem styles here based on the definition style (used for definitions and examples)
\newtheorem{definition}{Definition}

\theoremstyle{plain} % Define theorem styles here based on the plain style (used for theorems, lemmas, propositions)
\newtheorem{theorem}{Theorem}

\theoremstyle{remark} % Define theorem styles here based on the remark style (used for remarks and notes)

%----------------------------------------------------------------------------------------
%	HYPERLINKS
%---------------------------------------------------------------------------------------

\hypersetup{
%draft, % Uncomment to remove all links (useful for printing in black and white)
colorlinks=true, breaklinks=true, % bookmarks=true,
bookmarksnumbered,
urlcolor=webbrown, linkcolor=RoyalBlue, citecolor=webgreen, % Link colors
pdftitle={}, % PDF title
pdfauthor={\textcopyright}, % PDF Author
pdfsubject={}, % PDF Subject
pdfkeywords={}, % PDF Keywords
pdfcreator={pdfLaTeX}, % PDF Creator
plainpages=false,
pdfproducer={LaTeX with hyperref and ClassicThesis} % PDF producer
}


\renewcommand{\lstlistingname}{Python code}% Listing -> Python
\renewcommand{\lstlistlistingname}{List of \lstlistingname}% List of Listings -> List of Python
\definecolor{Code}{rgb}{0,0,0}
\definecolor{Decorators}{rgb}{0.5,0.5,0.5}
\definecolor{Numbers}{rgb}{0.5,0,0}
\definecolor{MatchingBrackets}{rgb}{0.25,0.5,0.5}
\definecolor{Keywords}{rgb}{0,0,1}
\definecolor{self}{rgb}{0,0,0}
\definecolor{Strings}{rgb}{0,0.63,0}
\definecolor{Comments}{rgb}{0,0.63,1}
\definecolor{Backquotes}{rgb}{0,0,0}
\definecolor{Classname}{rgb}{0,0,0}
\definecolor{FunctionName}{rgb}{0,0,0}
\definecolor{Operators}{rgb}{0,0,0}
\definecolor{Background}{rgb}{0.98,0.98,0.98}




 
\lstdefinestyle{mystyle}{
numbers=left,
numberstyle=\footnotesize,
numbersep=1em,
xleftmargin=1em,
framextopmargin=10em,
framexbottommargin=2em,
showspaces=false,
showtabs=false,
showstringspaces=false,
frame=l,
tabsize=4,
% Basic
basicstyle=\ttfamily\small\setstretch{1},
backgroundcolor=\color{Background},
language=Python,
% Comments
commentstyle=\color{Comments}\slshape,
% Strings
stringstyle=\color{Strings},
morecomment=[s][\color{Strings}]{"""}{"""},
morecomment=[s][\color{Strings}]{'''}{'''},
% keywords
morekeywords={import,from,class,def,for,while,if,is,in,elif,else,not,and,or,print,break,continue,return,True,False,None,access,as,,del,except,exec,finally,global,import,lambda,pass,print,raise,try,assert},
keywordstyle={\color{Keywords}\bfseries},
% additional keywords
morekeywords={[2]@invariant},
keywordstyle={[2]\color{Decorators}\slshape},
emph={self},
emphstyle={\color{self}\slshape},
captionpos=b, 
}
 
\lstset{style=mystyle}

\def\vl{\\[9pt]}
 % Specifies the document structure and loads requires packages

%----------------------------------------------------------------------------------------
%	ARTICLE INFORMATION
%----------------------------------------------------------------------------------------

\title{Quantifying cyclist behavior at intersections using video analysis} % The article title

\author{
	\authorstyle{Edi Bergovic\textsuperscript{1} and Høgni Jacobsen\textsuperscript{1}} % Authors
	\newline\newline % Space before institutions
	\textsuperscript{1}\institution{The IT University of Copenhagen, Copenhagen, Denmark}\\ % Institution 1
%	\textsuperscript{2}\institution{University of Texas at Austin, Texas, United States of America}\\ % Institution 2 
}

% Example of a one line author/institution relationship
%\author{\newauthor{John Marston} \newinstitution{Universidad Nacional Autónoma de México, Mexico City, Mexico}}

\date{\today} % Add a date here if you would like one to appear underneath the title block, use \today for the current date, leave empty for no date

%----------------------------------------------------------------------------------------

\begin{document}

\maketitle % Print the title

\thispagestyle{firstpage} % Apply the page style for the first page (no headers and footers)

%----------------------------------------------------------------------------------------
%	ABSTRACT
%----------------------------------------------------------------------------------------

% \lettrineabstract{Abstract}
\textbf{Abstract}

%----------------------------------------------------------------------------------------
%	ARTICLE CONTENTS
%----------------------------------------------------------------------------------------

\section{Introduction}

State what your research/project/inquiry is about. What are you writing about, why and for whom? What are your objectives? 
What are you trying to show or prove (your hypothesis)? Test cite \citep{Reference1}


\section{Methodology}

The Dybbølsbro intersection in Copenhagen was chosen as the location for our primary data collection. 
The Dybbølsbro intersection faces several traffic flow challenges as a result development in the immediate vicinity, and it is a large intersection.
These challenges make the Dybbølsbro intersection one of the more extreme in Copenhagen and would serve as a good base to this quantitative analysis method. 

To determine the desire paths that cyclist take throughout the Dybbelsbro intersection we recorded digital 2 hours of video footage 
at the Dybbølsbro intersection from three different camera angles.
The considerations taken in choosing a camera angle were:

\begin{itemize}
	\item Camera visibility to cyclists.
	\item Adequate mounting points, in terms of height and surface.
	\item Special attention was also given to making sure that cameras were not mounted on traffic signage.
\end{itemize}

\subsection{Point projection}
The video footage was analyzed using OpenDataCam which is an abstraction layer on top of Yolo. Yolo being an object detection library for object detection in images.
Once the video is analyzed by OpenDataCam, we receive a .json file containing a Unique ID for each unique cyclist that is detected in a frame of the video file. 
The unique ID is accompanied by an x and y coordinate of the detected bicycle on the frame. The (x, y) coordinates over multiple frames represents the track of an identified bicycle over the intersection.

\ \\
By assuming the road as a 2D plane, hereby ignoring any non-linear deformations (e.g. from lens distortion or curvature of the pavement), 
we can transform the pixel positions from the video to real-world 2D coordinates. 
We calculate the \textit{homography matrix}, describing the transformation from one plane to another, by mapping four reference points from each frame.

\begin{tabular}{cc}
\noindent
	\includegraphics[width=1.0\columnwidth]{projection_figure} 
\end{tabular}
\captionof{figure}{Reference points on map projection}

% To (x, y) coordinates are on the vertical plane of the video. To gain insight into the desire paths of the cyclists we projected the (x, y) coordinate down to the horizontal plane(the street) of the image.
% To do this we used the equation: 
% \begin{equation}
% 	y = y + h ⁄ 2
% \end{equation}

\subsection{Warped perspective}
With the tracks projected onto the roads surface we now warp (x, y) points in order to get a top-down view of the cyclist tracks.
This is achieved by calculating the homography matrix (H) between the source image and the destination image. The homography matrix being a 

\section{Results}

Give the results of your research. Do not, at this stage, try to interpret the results – simply report them. 
This section may include graphs, charts, diagrams etc. (clearly labelled). Be very careful about copyright if you are using published charts, tables, illustrations etc.


\section{Discussion}

Interpret your findings. What do they show? Were they what you expected? 
Could your research have been done in a better way?

\section{Conclusion}

These should follow on logically from the Findings and Discussion sections. Summarize the key points of your findings and show whether they prove or disprove your hypothesis. 
If you have been asked to, you can make recommendations arising from your research.

%----------------------------------------------------------------------------------------
%	BIBLIOGRAPHY
%----------------------------------------------------------------------------------------

\printbibliography[title={Bibliography}] % Print the bibliography, section title in curly brackets


%----------------------------------------------------------------------------------------

\end{document}