%%%%%%%%%%%%%%%%%%%%%%%%%%%%%%%%%%%%%%%%%
% Wenneker Article
% LaTeX Template
% Version 2.0 (28/2/17)
%
% This template was downloaded from:
% http://www.LaTeXTemplates.com
%
% License:
% CC BY-NC-SA 3.0 (http://creativecommons.org/licenses/by-nc-sa/3.0/)
%
%%%%%%%%%%%%%%%%%%%%%%%%%%%%%%%%%%%%%%%%%

%----------------------------------------------------------------------------------------
%	PACKAGES AND OTHER DOCUMENT CONFIGURATIONS
%----------------------------------------------------------------------------------------

\documentclass[10pt, a4paper, twocolumn]{article} % 10pt font size (11 and 12 also possible), A4 paper (letterpaper for US letter) and two column layout (remove for one column)

%%%%%%%%%%%%%%%%%%%%%%%%%%%%%%%%%%%%%%%%%
% Arsclassica Article
% Structure Specification File
%
% This file has been downloaded from:
% http://www.LaTeXTemplates.com
%
% Original author:
% Lorenzo Pantieri (http://www.lorenzopantieri.net) with extensive modifications by:
% Vel (vel@latextemplates.com)
%
% License:
% CC BY-NC-SA 3.0 (http://creativecommons.org/licenses/by-nc-sa/3.0/)
%
%%%%%%%%%%%%%%%%%%%%%%%%%%%%%%%%%%%%%%%%%

%----------------------------------------------------------------------------------------
%	REQUIRED PACKAGES
%----------------------------------------------------------------------------------------

\usepackage[
%nochapters, % Turn off chapters since this is an article        
beramono, % Use the Bera Mono font for monospaced text (\texttt)
eulermath,% Use the Euler font for mathematics
pdfspacing, % Makes use of pdftex letter spacing capabilities via the microtype package
dottedtoc % Dotted lines leading to the page numbers in the table of contents
]{classicthesis} % The layout is based on the Classic Thesis style


\usepackage{arsclassica} % Modifies the Classic Thesis package

\usepackage[T1]{fontenc} % Use 8-bit encoding that has 256 glyphs

\usepackage[utf8]{inputenc} % Required for including letters with accents

\usepackage{graphicx} % Required for including images

\usepackage{enumitem} % Required for manipulating the whitespace between and within lists

\usepackage{lipsum} % Used for inserting dummy 'Lorem ipsum' text into the template

\usepackage{subfig} % Required for creating figures with multiple parts (subfigures)

\usepackage{amsmath,amssymb,amsthm} % For including math equations, theorems, symbols, etc

\usepackage{varioref} % More descriptive referencing
\usepackage{color}
\usepackage{listings}
\usepackage{setspace}
\usepackage{lscape}
\usepackage[authoryear]{natbib}

%----------------------------------------------------------------------------------------
%	DUTCH SUPPORT
%----------------------------------------------------------------------------------------

%\usepackage[dutch]{babel}    % comment out if you write your thesis in Dutch


%----------------------------------------------------------------------------------------
%	THEOREM STYLES
%---------------------------------------------------------------------------------------

\theoremstyle{definition} % Define theorem styles here based on the definition style (used for definitions and examples)
\newtheorem{definition}{Definition}

\theoremstyle{plain} % Define theorem styles here based on the plain style (used for theorems, lemmas, propositions)
\newtheorem{theorem}{Theorem}

\theoremstyle{remark} % Define theorem styles here based on the remark style (used for remarks and notes)

%----------------------------------------------------------------------------------------
%	HYPERLINKS
%---------------------------------------------------------------------------------------

\hypersetup{
%draft, % Uncomment to remove all links (useful for printing in black and white)
colorlinks=true, breaklinks=true, % bookmarks=true,
bookmarksnumbered,
urlcolor=webbrown, linkcolor=RoyalBlue, citecolor=webgreen, % Link colors
pdftitle={}, % PDF title
pdfauthor={\textcopyright}, % PDF Author
pdfsubject={}, % PDF Subject
pdfkeywords={}, % PDF Keywords
pdfcreator={pdfLaTeX}, % PDF Creator
plainpages=false,
pdfproducer={LaTeX with hyperref and ClassicThesis} % PDF producer
}


\renewcommand{\lstlistingname}{Python code}% Listing -> Python
\renewcommand{\lstlistlistingname}{List of \lstlistingname}% List of Listings -> List of Python
\definecolor{Code}{rgb}{0,0,0}
\definecolor{Decorators}{rgb}{0.5,0.5,0.5}
\definecolor{Numbers}{rgb}{0.5,0,0}
\definecolor{MatchingBrackets}{rgb}{0.25,0.5,0.5}
\definecolor{Keywords}{rgb}{0,0,1}
\definecolor{self}{rgb}{0,0,0}
\definecolor{Strings}{rgb}{0,0.63,0}
\definecolor{Comments}{rgb}{0,0.63,1}
\definecolor{Backquotes}{rgb}{0,0,0}
\definecolor{Classname}{rgb}{0,0,0}
\definecolor{FunctionName}{rgb}{0,0,0}
\definecolor{Operators}{rgb}{0,0,0}
\definecolor{Background}{rgb}{0.98,0.98,0.98}




 
\lstdefinestyle{mystyle}{
numbers=left,
numberstyle=\footnotesize,
numbersep=1em,
xleftmargin=1em,
framextopmargin=10em,
framexbottommargin=2em,
showspaces=false,
showtabs=false,
showstringspaces=false,
frame=l,
tabsize=4,
% Basic
basicstyle=\ttfamily\small\setstretch{1},
backgroundcolor=\color{Background},
language=Python,
% Comments
commentstyle=\color{Comments}\slshape,
% Strings
stringstyle=\color{Strings},
morecomment=[s][\color{Strings}]{"""}{"""},
morecomment=[s][\color{Strings}]{'''}{'''},
% keywords
morekeywords={import,from,class,def,for,while,if,is,in,elif,else,not,and,or,print,break,continue,return,True,False,None,access,as,,del,except,exec,finally,global,import,lambda,pass,print,raise,try,assert},
keywordstyle={\color{Keywords}\bfseries},
% additional keywords
morekeywords={[2]@invariant},
keywordstyle={[2]\color{Decorators}\slshape},
emph={self},
emphstyle={\color{self}\slshape},
captionpos=b, 
}
 
\lstset{style=mystyle}

\def\vl{\\[9pt]}
 % Specifies the document structure and loads requires packages
\usepackage{subfiles} % Best loaded last in the preamble
\usepackage{setspace}

\hyphenation{intersections OpenDataCam}

% Abstract type defination
\renewenvironment{abstract}
 {\small
  \begin{center}
  \bfseries\vspace{-.5em}\vspace{0pt}
  \end{center}
  \list{}{
    \setlength{\leftmargin}{3.0cm}%
    \setlength{\rightmargin}{\leftmargin}%
  }%
  \item\relax}
 {\endlist}


%----------------------------------------------------------------------------------------
%	ARTICLE INFORMATION
%----------------------------------------------------------------------------------------

\title{
	Quantifying cyclist behavior at intersections using automated video analysis 
	\ \\
	} % The article title

\author{
	\begin{spacing}{1.7}
	\authorstyle{Edi Begovic (edbe@itu.dk) \\ Høgni Jacobsen (hoja@tiu.dk) } % Authors
	\newline
	\institution{Bachelor project}\\ % Institution
	\newline
	\authorstyle{Supervised by Michael Szell} % Authors
	\\ \\ \\
	\includegraphics[width=3in]{img/ITUlogo} 
	\end{spacing}
}

\date{\today} % Add a date here if you would like one to appear underneath the title block, use \today for the current date, leave empty for no date

%----------------------------------------------------------------------------------------
%	TITLE PAGE
%----------------------------------------------------------------------------------------

\begin{document}

\maketitle % Print the title
\thispagestyle{empty}  % Apply the page style for the first page (no headers and footers)

%----------------------------------------------------------------------------------------
%	ABSTRACT
%----------------------------------------------------------------------------------------

\twocolumn[\begin{@twocolumnfalse}
\ \\ \\
\begin{abstract}

\centerline{\LARGE \textbf{Abstract}} 

\ \\
In many European cities, such as Copenhagen and Amsterdam, cycling is an essential part of urban 
transportation. Transport infrastructure often prioritizes cars, resulting in unfriendly or even dangerous 
environments for cycling. 
Having quantifiable data on the movement and behavior of cyclists is a crucial element in modern city
planning and would result in friendlier urban infrastructure for cyclists. The current data
collection methods are reliant on the human annotation of video recordings, which is expensive, time-consuming,
and could face challenges in reproducibility. Here we set up and evaluate a new methodological approach to replace 
manual annotation.
In this approach, we set up a data collection and analysis pipeline consisting of camera selection and setup, 
video syncing, camera calibration, object detection, projection, merging of different camera sources, and object tracking. 
We also develop a web app for efficiently exploring collected footage. 
To test our setup in a case study, we record and analyze videos of the Dybbelsbro intersection in Copenhagen 
with multiple cameras. Our method succeeds in detecting cyclist trajectories and transforming them into a 
birds-eye view, allowing us to explore both fine-grained details and an aggregated view of cyclist behavior.
Using our method, we produce favorable results compared to manual annotation.

\end{abstract}
\end{@twocolumnfalse}]
\thispagestyle{empty}
\clearpage

%----------------------------------------------------------------------------------------
%	Table of Contents 
%----------------------------------------------------------------------------------------
\twocolumn[\begin{@twocolumnfalse}
\setcounter{tocdepth}{2} % Set the depth of the table of contents to show sections and subsections only
\tableofcontents % Print the table of contents
\end{@twocolumnfalse}]
\thispagestyle{empty}
\clearpage
%----------------------------------------------------------------------------------------
%	ARTICLE CONTENTS
%----------------------------------------------------------------------------------------

% Page numbering starts from here
\setcounter{page}{1}

\section{Introduction}
\subfile{sections/introduction}

\section{Background}
\subfile{sections/background}

\section{Methodological proposal}
\subfile{sections/proposal}

\section{Implementation}
\subfile{sections/methodology}

\section{Case studies}
\subfile{sections/results}

\section{Discussion}
\subfile{sections/discussion}

\section{Conclusion}
\subfile{sections/conclusion}

%----------------------------------------------------------------------------------------
%	BIBLIOGRAPHY
%----------------------------------------------------------------------------------------
\clearpage
\printbibliography[title={Bibliography}] % Print the bibliography, section title in curly brackets


%----------------------------------------------------------------------------------------
%   APPENDIX
%----------------------------------------------------------------------------------------

\onecolumn

\begin{appendices}
\section{Behavior analysis}
\includegraphics[scale=0.2]{confused.jpg}

\section{Undistorted pictures}
\includegraphics[scale=0.56]{Undistort.jpg} 

\end{appendices}

% ---------------------------------------------------------------------------------------
\end{document}